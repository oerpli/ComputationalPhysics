% !TeX root = ../Harte_Kugeln.tex
Im Gegensatz zur herkömmlichen Molekulardynamik bei der die Bewegung der Moleküle aus Integration der wirkenden Kräfte mit einem konstanten Zeitschritt folgt wird bei der ereignisbasierten Molekulardynamik ein Index über alle (zumindest alle relevanten) in der Zukunft liegenden Stöße geführt. Nach jedem Stoß wird die Zeit bis zum nächsten Stoß ``vorgespult'', eine Kollision berechnet und der Index aktualisiert.

Das untersuchte System war eine Box mit periodischen Randbedingungen gefüllt mit harten Kugeln -- solange die Kugel sich frei bewegen kann wirkt kein Potential, nur durch Stöße an anderen Kugeln erfolgt eine Änderung von Impuls und Energie. Zusätzlich werden die Stöße als perfekt elastisch simuliert, das heißt, dass keine Energie durch Zustandsänderung der Kugeln zugeführt wird oder verloren geht (keine Deformation, Erwärmung der Kugeln, auch kein Drehimpuls). 

Die Motivation der Methode ist, dass bei herkömmlicher MD (Molekulardynamik) mit konstantem Zeitschritt zwischen zwei Stößen viele unnötige Berechnungen durchgeführt werden. Wenn man den Zeitschritt größer macht würde sich diese Anzahl zwar senken, dafür hätte man bei Stößen das Problem, dass sich eine Kugel innerhalb eines Zeitschritts in eine andere hineinbewegen könnte was dann entweder zu Ungenauigkeiten oder zusätzlichem Rechenaufwand (Zeit zurück drehen und mit kleineren Zeitschritten sondieren) führt.