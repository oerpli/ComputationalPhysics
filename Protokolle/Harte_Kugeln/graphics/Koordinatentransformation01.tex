% !TeX root = ../Harte_Kugeln.tex

\begin{figure}[H]
  \centering
  \begin{tikzpicture}
    \coordinate (Origin)   at (0,0);
    \coordinate (OL) at (0,\BoxH);
    \coordinate (OR) at (\BoxW,\BoxH);
    \coordinate (UR) at (\BoxW,0);
    
    \draw [very thick] (Origin) -- (OL) -- (OR) -- (UR) -- (Origin);
    
    \coordinate [label={below right:$K_2$}] (K2) at (\Kx,\Ky);
    \drawKugel{K2};
    
    \coordinate [label={below:$K_1$}] (K1) at (\Kx+\Dx,\Ky+\Dy);
    \drawKugel{K1};

	\draw[-latex, thick] (K2) -- (K1) node[midway,above,sloped] {$\vdis$};
	
	\draw[-latex, thick] (K1) -- (\Kx+\Dx+\vxa, \Ky+\Dy+\vya) node[midway,above,sloped] {$\vec{v}_1$};
	\draw[-latex, thick] (K2) -- (\Kx+\vxb, \Ky+\vyb) node[midway,above,sloped] {$\vec{v}_2$};
  \end{tikzpicture}
  \caption{Ausgansposition der betrachteten Kugeln.}
  \label{fig:Koortrans01}
\end{figure}



