In the course of this project, we managed to show that introducing a thermostat into a system of a single quantum para-hydrogen molecule modelled as a ring polymer of classical particles can overcome the problem of decoupled frequency modes in this harmonic oscillator. However, as we have seen, the choice of thermostat is crucial. Also, the behaviour of some of the thermostats used relies heavily on the values of their parameters --- which are not always a straight-forward choice. Especially for the Berendsen thermostat and it's relative, the Bussi--Donadio--Parinello thermostat, we did not find a set of parameters that lead to a reasonable distribution in temperature. 
On the other hand, we found that a simple thermostat like the Gaussian or the Andersen, yield good results even though the first represents an isokinetic ensemble and the second is not time-reversible. Overall, the momentum-conserving analogue to the Andersen thermostat --- the Lowe--Andersen thermostat --- lead to the strongest agreement with the theory and can therefore be recommended for further studies of path integral molecular dynamics.    