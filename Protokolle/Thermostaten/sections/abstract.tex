% !TeX root = .
The following work was produced in the course of the Laboratory on Computational Physics in the winter term of 2014 at the University of Vienna under the supervision of Professor Martin Neumann. This work presents an overview and comparison of different thermostats used in molecular dynamics simulations in the canonical (NVT) ensemble. To test these thermostats, we used a system of quantum hydrogen modelled as a ring polymer consistent with the Path Integral Molecular Dynamics theory which exploits the isomorphism between quantum theory and classical statistical mechanics.
The rest of the paper is organized as follows. First, a short introduction on molecular dynamics simulations in the NVT ensemble will be given and the concepts of Path Integral Molecular Dynamics will be explained. Then, the theory and implementation of the thermostats under investigation will be introduced. This will be followed by a presentation of the results, a discussion of possible errors and problems and a conclusion on the advantages and disadvantages of the studied thermostats as well as their applicability and efficiency. 
