% !TeX root = ../main.tex

% Vllt als Konvention T die gewünschte Temperatur verwenden und T(t) die aktuelle Temperatur des Systems

Our basic simulation without thermostat enabled samples from a microcanonical ensemble with constant energy (isokinetic, NVE) -- this is given by the rules governing the interactions between particles.
To achieve that temperature instead of energy is conserved (isothermal) a thermostat is used. These usually interject some kind of rescaling or additional terms in the integration steps achieving slightly modified trajectories

We implemented the following thermostats: % vllt jeden thermostat in einem abschnitt beschreiben und hinterher die liste löschen. 
\begin{itemize}
\item Andersen (parameter that controls exchange rate between random particles and heat bath)
\item Lowe-Andersen
\item Berendsen (parameter that controls temperature change rate from actual to preferred temperature)
\item Bussi et al.
\item Gaussian (velocity rescaling -- fixes temperature $T(t) \equiv T$)
\item Nose-Hoover
\item Nose-Hoover Chains
\end{itemize}

\paragraph{Andersen:}
\paragraph{Lowe--Andersen}
\paragraph{Berendsen:}The temperature is tied to a heat-- bath and a new parameter is introduced that describes the termpature change rate $\tau$:
\begin{equation*}
\frac{dT(t)}{dt} = \frac{T-T(t)}{\tau} %die formel sieht ziemlich blöd aus finde ich grad - hat jmd eine idee für eine bessere konvention bzgl wunsch und aktueller temperatur?
\end{equation*}
A sane (\TODO) choice of this rate allows sampling of a canonical ensemble.
%expand me please

\paragraph{Bussi et al:}
\paragraph{Gaussian:} This thermostat rescales the velocities of the particles to match the desired temperature T. This is achieved by calculating the current temperature of the system (Eq. \eqref{eq:tempdef}) and rescaling the velocities $v$ according to the following formula:
%: temperaturberechnung in unserer simulation irgendwo definieren und hier referenzieren.
\begin{equation*}
v'  = v\cdot \sqrt{\frac{T}{T(t)}}
\end{equation*}
The problem of this approach is that it produces a constant temperature which is not entirely in line with the canonical ensemble because sampling from such an ensemble would lead to fluctuations in the temperature and a variance $\sigma_T^2 > 0$.

As this thermostat does not yield any known ensemble we were advised to not use it for serious projects, due to the simplicity of the approach it may still be worth considering for testing purposes or similar ventures.


\paragraph{Nose-Hoover:}
\paragraph{Nose-Hoover Chains:}


