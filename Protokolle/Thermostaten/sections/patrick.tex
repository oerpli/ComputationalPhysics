% !TeX root = ../Thermostats.tex
\subsection{Andersen}

The Andersen thermostat was first introduced by Hans Andersen in 1980\cite{Andersen1}.
It achieves to hold the designated energy by connecting the system to an external heat bath with the corresponding temperature T.
In every time step each particle has a probability $P=\nu\cdot\Delta t$, with $\nu$ the stochastic collision frequency, to undergo a collision with the heat bath and thus change the momentum. These collisions are instantaneous and effect only the one particle involved.
The new momentum of the particle after a collision is drawn at random from a Bolzmann contribution of the given temperature.
Because only a small, random number of particles are affected each time step most of the particles move freely according to the Hamiltonian. But the encounters with the heat bath are enough to relax the system to the given temperature and to let it fluctuate around its equilibrium according to the canonical ensemble.

The Andersen thermostat should only be applied to time-independent properties, dynamic problems should not be thermostated by an Andersen algorithm\cite{Andersen2}.

\subsection{Lowe--Andersen}
The Lowe-Andersen thermostat is a Galilean invariant and momentum conserving analogue of the Andersen thermostat\cite{LoweAndersen}.
The collisions with the heat bath now affect a pair of two particles, and only the relative velocity along the centre of mass changes. The relative velocity $v_{ij}$ from the Maxwell-Boltzmann distribution is calculated with the equation:
\begin{equation}
v_{ij}'=\zeta\sqrt{2k_BT},
\end{equation}
with $\zeta$ being a random number with unit variance and zero mean.
The relative velocity of the particles after the collision is then adjusted according to
\begin{align}
2\Delta_{ij}&=r_{ij}(v_{ij}'-v_{ij})\cdot r_{ij}\\
v_i'&=v_i+\Delta_{ij}\\
v_j'&=v_j-\Delta{ij}.\\
\end{align}
