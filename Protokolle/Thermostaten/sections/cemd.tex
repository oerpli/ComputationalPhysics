% !TeX root = ../Thermostats.tex
Molecular dynamics (MD) is a method to follow the time-evolution of an $N$-body system with the help of numerical integration. Given that the system under investigation is ergodic and thus time averages are equivalent to ensemble averages, it is possible with such a simulation to obtain statistical information about various properties of the system - such as kinetic and potential energy, radius of gyration, pair-correlation functions, etc. 
A classical molecular dynamics simulation works as follows. The system is set up with initial coordinates and velocities for each particle, which are usually chosen such that the desired temperature is obtained and the total momentum is zero. Next, the forces acting on each particle are calculated. Once all forces are known, Newton's equations of motion can be integrated to propagate the system in time. The last two steps are repeated for as long as necessary to get good statistical averages. 
In most MD simulations, the force calculation is the most time-consuming step, since interactions of the order of $N^2$ have to be calculated. Various algorithms have been devised to reduce the cost of this step, for example cell lists or Verlet lists, where only pair interactions of close particles are calculated at each step. 
A variety of integrating algorithms exist and one has to find a balance between precision and computational cost. Some of the most important aspects to consider are time-reversibility, long-term energy drift and conservation of phase-space volume. The velocity-verlet algorithm is often used for these reasons, even though it exhibits moderate short-time fluctuations of the total energy. The update of the positions and velocities looks as follows: 
\begin{align*}
& r(t + \Delta t) = r(t) + v(t)\Delta t + \frac{F(t)}{2m}\Delta t^2 \\
& v(t + \Delta t) = v(t) + \frac{F(t+\Delta t) + F(t)}{2m} \Delta t \numberthis
\end{align*}     

Note that in this formula, it looks as if the force on each particle would have to be calculated twice at  each step. This is not true, however. The calculated values of $F(t+\Delta t)$ needed for the velocity update at time $t+\Delta t$ can be used for the position update at time $t+2\Delta t$, so only one extra force calculation at the beginning of the simulation is needed. 

The scheme discussed so far corresponds to a microcanonical ($NVE$) ensemble, in which the total energy is a constant of motion. However, one is often more interested in simulating systems in a different ensemble, for example in the canonical ensemble ($NVT$), which is also the main focus of this project. In the canonical ensemble, the temperature $T$ is constant and the total energy $E$ is allowed to fluctuate.  This corresponds to the situation in which the system is embedded in a larger system - the so-called 'heat bath' - with which it constantly exchanges energy. The various methods of implementing a canonical molecular dynamics simulation will be covered in section \ref{thermostats}.  